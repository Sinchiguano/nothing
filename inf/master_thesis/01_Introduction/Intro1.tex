\chapter{Introduction}
\label{chap:intro}

Within this chapter, the reader receives an outline of the general context which surrounds this thesis. Starting with the motivation section and the ultimate goal to be accomplished, and a summary of the thesis' structure follow. 
%And finally, an overview of related work is presented.

\section{Motivation}


For years, The industrial robot has undergoes through enormous development. Robot nowadays not only receives command from the computer. But also has the ability to make decision itself. Such abilities are well known in the world of the computer vision as recognizing and determining 6D pose of a rigid body (3D translation and 3D rotation). However, finding the object of interest or determining its pose in either 2D or 3D scenes is still a challenging task for computer vision. There are many researchers working on it with method that goes from state-of-the-art to deep learning means where the object is usually represented with a CAD model or object's 3D reconstruction and typical task is detection of this particular object in the scene captured with RBGD or depth camera. Detection consider determining the location of the object in the input image. This is typical in robotics and machine vision applications where the robot usually does task like pick and place objects. However, localization and pose estimation is much more challenging task due to the high dimensionality of the search in the workspace. And addition, the object of interest is often sought in cluttered scenes under occlusion with requirement of real-time performance.

\section{Goal}


We attempt to provide an algorithm for the localization of the known parts(6D object pose estimation using RGBD data) \cite{intro2}. In addition, a robot-camera calibration needs to be done, and a 3D object model is created.

\section{Thesis structure}


The next chapter gives a brief overview of the field of computer vision and related work. We have a look at the classical way the community tackle the problem of pose estimation of rigid body. This is followed by a description of the ROS which stand for Robot operating system, and a background information in order to keep up with the method we will implement.



\begin{figure}[h]
\begin{center}
\includegraphics[height=7cm]{figures01/linux1.jpeg}
\caption{An example of the social event on a webpage}
\label{fig:webevent}
\end{center}
\end{figure}

