We gaan een tijd tegemoet waarin robots niet enkel meer achter veiligheidskooien werken. Robots werken nu meer en meer in een interactie met mensen. De robots die we gebruiken voor deze robot-mens interactie worden collaboratieve robots of cobots genoemd. Een nieuwe technologie die nu beschikbaar is om deze cobots veiliger en vriendelijker te maken is force compliance. En dat is het onderwerp waar dit proefschrift over gaat, force compliant robots en hun indrukwekkende functies. Door de force feedback zullen dit soort robots de robot-menselijke interactie naar een hoger niveau tillen. Mensvriendelijke manipulatie van de robot en een vlotte reactie van de robot op botsingen zijn mogelijk. In mijn proefschrift heb ik een aantal applicaties ontwikkeld om zo goed mogelijk te laten zien wat deze robot kan. Ik dook in de programmering van de robot en in zijn positie- en koppelregeling om een ​​goede kijk te hebben op zijn mogelijkheden. De toepassingen die ik heb ontwikkeld en die in dit proefschrift worden beschreven zijn: Een voorwerp aan de robot geven, de robot als een derde hand gebruiken, de robot duwen met een zachte bal en de robot de rug van een persoon laten masseren. Deze applicaties laten zeer goed zien wat deze force-compliance inhoudt. Het is een enorm uitgestrekt en interessant gebied waar nog steeds veel ruimte is voor creativiteit en innovatie. Voor het onderzoek en de tests heb ik een KUKA LBR iiwa force compliant robot gebruikt en het Robot Operating System (ROS) op Linux Ubuntu 16.04 LTS.\\

\textbf{Keywords}: Cobots, compliance, KUKA, ROS, applicaties
