We are entering a time where robots do not only work behind safety cages anymore. Robots are now working more and more in an interaction with humans. The robots we use for this robot-human interaction are called collaborative robots or cobots. A new technology which is available now to make these cobots safer and friendlier to work with is called force compliance. And that's the subject where this thesis is about, force compliant robots and their impressive functions. Due to this force feedback, this kind of robots will take the robot-human interaction to a higher level. Soft robot handling and soft reaction of the robot to collisions are possible. In my thesis, I developed some applications to show as good as possible what this robot is able to. I dived into the programmation of the robot and into its position and force control to have a very good understanding of its capabilities. The applications that I developed and that are described in this thesis are: Giving an object to the robot, use the robot as a third hand, pushing the robot with a balloon and let the robot scratch a persons back. These applications show very well what this force compliance is up to. It is a really vast and exciting area where there is still a lot of place for creativity and innovation. For the research and tests, I used a KUKA LBR iiwa force compliant robot and the Robot Operating System (ROS) on Linux Ubuntu 16.04 LTS.\\

\textbf{Keywords}: Cobots, compliance, KUKA, ROS, applications

