\documentclass[10]{article}
\begin{document}
\input ctustyle2  % The template (in version 2) is included here.
%\input pdfuni    % Uncomment this if you need accented PDFoutlines
%\input opmac-bib % Uncomment this for direct reading of .bib database files 

\worktype [O/EN] % Type: B = bachelor, M = master, D = Ph.D., O = other
                 % / the language: CZ = Czech, SK = Slovak, EN = English

\faculty    {F3}  % Type your faculty F1, F2, F3, etc. or MUVS
            % use main language of your document here:
\department {Katedra matematiky}
\title      {CTUstyle -- návod k použití}
\subtitle   {Šablona v plain\TeX{}u\nl pro sazbu 
             studentských závěrečných prací na ČVUT}
            % \subtitle is optional
\author     {Petr Olšák}
\date       {Leden 2013}
\supervisor {}  % One or more supervisors
\studyinfo  {}  % Study programme etc.
\workname   {Dokumentace} % Used only if \worktype [O/*] (Other)
            % optional more information about the document:
\workinfo   {\url{http://petr.olsak.net/ctustyle.html}}
            % Title / Subtitle in minor language:
\titleEN    {CTUstyle -- the user manual}
\subtitleEN {the plain\TeX{} template for theses at CTU}
            % If minor language is other than English
            % use \titleCZ, \subtitleCZ or \titleSK, \subtitleSK instead it.
\pagetwo    {}  % The text printed on the page 2 at the bottom.

\abstractEN {
   This document shows and tests an usage of the plain\TeX{} officially
   (may be) recommended design style {\ssr CTUstyle} for bachelor (Bsc.), master
   (Ing.), or doctoral (Ph.D.) theses at the Czech Technical University in
   Prague. The template defines all thesis mandatory structural elements and
   typesets their content to fulfil the university formal rules.

   This is version 2 of this template which implements the Technika font
   recommended by CTU graphics identity reference since 2016.
}
\abstractCZ {
   Tento dokument ukazuje a testuje použití (možná) oficiálně doporučené 
   plain\TeX{}ové šablony {\ssr CTUstyle} pro sazbu bakalářských, 
   diplomových a disertačních prací na Českém vysokém učení technickém v~Praze. 
   Šablona definuje všechny povinné strukturní
   elementy zmíněných závěrečných prací a formátuje jejich obsah tak, aby
   splňovala na škole daná formální pravidla.

   Toto je verze 2 této šablony, která na rozdíl od předchozí verze
   implementuje písmo Technika v souladu s doporučením manuálu vizuálního
   stylu, který zavedlo ČVUT v roce 2016.
}           % If your language is Slovak use \abstractSK instead \abstractCZ

\keywordsEN {%
   document design template; bachelor, master, Ph.D. thesis; \TeX{}.
}
\keywordsCZ {%
   styl dokumentu; šablona; 
   bakalářská, diplomová, disertační závěrečná práce; \TeX{}.
}
\thanks {           % Use main language here
   Chtěl bych poděkovat své manželce Ludmile za podporu nejen finanční.
   Díky tomu mohu na svém pracovišti dělat, co mě baví, a nejsem stresován 
   výplatní páskou.
}
\declaration {      % Use main language here
   Prohlašuji, že jsem předloženou práci vypracoval
   samostatně a že jsem uvedl veškeré použité informační zdroje v~souladu
   s~Metodickým pokynem o~dodržování etických principů při přípravě
   vysokoškolských závěrečných prací.

   V Praze dne 13. 13. 2013 % !!! Attention, you have to change this item.
   \signature % makes dots
}

%%%%% <--   % The place for your own macros is here.

%\draft     % Uncomment this if the version of your document is working only.
%\linespacing=1.7  % uncomment this if you need more spaces between lines
                   % Warning: this works only when \draft is activated!
%\savetoner        % Turns off the lightBlue backround of tables and
                   % verbatims, only for \draft version.
%\blackwhite       % Use this if you need really Black+White thesis.
%\onesideprinting  % Use this if you really don't use duplex printing. 

\makefront  % Mandatory command. Makes title page, acknowledgment, contents etc.

\input uvod    % Files where the source of the document is prepared.
\input popis   % Full name is: uvod.tex, popis.tex, the suffix can be omitted.
\input prilohy

\bye

\end{document}